\documentclass[11pt,a4paper]{article}

% Packages
\usepackage[utf8]{inputenc}
\usepackage[margin=1in]{geometry}
\usepackage{graphicx}
\usepackage{amsmath}
\usepackage{hyperref}
\usepackage{float}
\usepackage{subcaption}
\usepackage{listings}
\usepackage{color}
\usepackage{booktabs}


\definecolor{dkgreen}{rgb}{0,0.6,0}
\definecolor{gray}{rgb}{0.5,0.5,0.5}
\definecolor{mauve}{rgb}{0.58,0,0.82}
\lstset{frame=tb,
  language=python,
  aboveskip=3mm,
  belowskip=3mm,
  showstringspaces=false,
  columns=flexible,
  basicstyle={\small\ttfamily},
  numbers=none,
  numberstyle=\tiny\color{gray},
  keywordstyle=\color{blue},
  commentstyle=\color{dkgreen},
  stringstyle=\color{mauve},
  breaklines=true,
  breakatwhitespace=true,
  tabsize=3
}

\title{Lab 2: Energy Efficient Displays}
\author{Your Name \\ Student ID}
\author{Your Name \\ Student ID}

\date{\today}

\begin{document}


\maketitle

\section{Part 1: Energy Efficient Image Processing}

\subsection{Implementation}



\begin{figure}[h]
\centering
\begin{lstlisting}
 Example Code
\end{lstlisting}
\caption{Transformation function definitons}
\label{fig:transformation_defs}
\end{figure}


\graphicspath{{images}}
\begin{figure}[H]
  \centering
  %\includegraphics[width=0.7\textwidth]{Pareto_of_default_image_provided.png}
  \caption{Pareto graph of different image transformation techniques}
  \label{fig:pareto_graph1}
\end{figure}



\begin{table}[htbp]
\centering
\caption{Power saving statistics for each transformation}
\label{tab:power_saving_stats}
\begin{tabular}{lrrr}
\toprule
\textbf{Transform} &
\textbf{Mean PS (\%)} &
\textbf{Min PS (\%)} &
\textbf{Max PS (\%)} \\
\midrule
Bright scale & 17.18 & 16.85 & 18.46 \\
Histogram    & -5.53 & -135.09 & 30.56 \\
Hungry blue  & 6.43 & 2.78 & 20.24 \\
\bottomrule
\end{tabular}
\end{table}


\end{document}