\documentclass[11pt,a4paper]{article}

% Packages
\usepackage[utf8]{inputenc}
\usepackage[margin=1in]{geometry}
\usepackage{graphicx}
\usepackage{amsmath}
\usepackage{hyperref}
\usepackage{float}
\usepackage{subcaption}
\usepackage{listings}
\usepackage{color}
\usepackage{booktabs}
\usepackage[utf8]{inputenc}
\usepackage[numbers]{natbib} 




\definecolor{dkgreen}{rgb}{0,0.6,0}
\definecolor{gray}{rgb}{0.5,0.5,0.5}
\definecolor{mauve}{rgb}{0.58,0,0.82}
\lstset{frame=tb,
  language=python,
  aboveskip=3mm,
  belowskip=3mm,
  showstringspaces=false,
  columns=flexible,
  basicstyle={\small\ttfamily},
  numbers=none,
  numberstyle=\tiny\color{gray},
  keywordstyle=\color{blue},
  commentstyle=\color{dkgreen},
  stringstyle=\color{mauve},
  breaklines=true,
  breakatwhitespace=true,
  tabsize=3
}

\title{GPU-Accelerated LIF Spiking Neuron Networks Simulator}
\author{Your Name \\ Student ID}


\date{\today}

\begin{document}


\maketitle

\section{Introduction}

The Project is done to implement more compact and lightweight 
SNN simulator compared to the ones already available. 

The neuron implementation is numerical instead of analog. 
Firstly, the neuron is calculated, 
then the value being comapared to the threshold voltage, and resulting a spike. 
The refractory phase will start for the spiked neuron.
Lastly, the synaptic updates are being calculated as exponential decay per timestep. 



\section{Implementation}
\subsection{LIF}
The discrete-time implementation of the Leaky Integrate-and-Fire (LIF) model 
as discussed by Stan and Rhodes \cite{stan2024learning} allows for 
efficient sequence modeling in SNNs.

The parameters in the equations are: \\

\begin{tabular}{r l}
  $u[t]$ & Membrane voltage (potential) at time step $t$. \\[0.5em]
  $\beta$ & Membrane decay factor. \\[0.5em]
  $s[t]$ & Spike indicator. \\[0.5em]
  $\theta$ & Firing threshold voltage. \\[0.5em]
  $i[t]$ & Input current.
\end{tabular}
\\

The membrane potential update is being done with the equation~\ref{eq:lif_integration}. 

The Reset mechanism is given in equation~\ref{eq:lif_reset} as soft reset. 
Instead of resetting to a default value, 
the membrane voltage is being calculated directly.
This prevent warp divergence if the hard reset chosen, 
with condition structure will be introduced.

Lastly, spike generation is being calculated with equation~\ref{eq:lif_spike}. 
The spike is being decided with comparison between membrane voltage and threshold.

% Membrane Potential Update (Integration)
\begin{equation}
u[t] = \beta u[t-1] + (1 - \beta) i[t]
\label{eq:lif_integration}
\end{equation}

% Reset Mechanism (Soft Reset)
\begin{equation}
u[t] \leftarrow u[t] - s[t-1]\theta
\label{eq:lif_reset}
\end{equation}

% Spike Generation (Heaviside Step Function)
\begin{equation}
s[t] = \Theta(u[t] - \theta) = 
\begin{cases} 
1, & \text{if } u[t] > \theta \\
0, & \text{otherwise}
\end{cases}
\label{eq:lif_spike}
\end{equation}

% Definition of Beta (Decay Factor)
where the decay factor $\beta$ is defined by the membrane time constant $\tau$ and simulation time step $\Delta t$:
\begin{equation}
\beta = \exp\left(-\frac{\Delta t}{\tau}\right)
\end{equation}



\subsection{Synapse Model}
The synapse model is decided as current-based exponential synapse model.
The equation of the exponential synapse model is given in the equation~\ref{eq:discrete_synapse}.\\

Parameters are given for synapse model: \\[0.5em]
\begin{tabular}{r l}
  $\alpha$ & Synaptic decay factor ($e^{-\Delta t / \tau_s}$). \\[0.5em]
  $w_{ji}$ & Weight from neuron $j$ to $i$. \\[0.5em]
  $s_j[t]$ & Spike indicator from presynaptic neuron $j$. \\[0.5em]
  $g_i[t]$ & Postsynaptic current state (decaying memory). \\[0.5em]
  $u_i[t]$ & Membrane potential of neuron $i$. \\[0.5em]
  $\beta$ & Membrane decay factor ($e^{-\Delta t / \tau_m}$). \\[0.5em]
  $\theta$ & Spike threshold.
\end{tabular}


\begin{equation}
g_i[t] = \alpha\, g_i[t-1] + \sum_j w_{ji}\, s_j[n]
\label{eq:discrete_synapse}
\end{equation}

Connecting this synapse model with the neuron model is 
resulted this equation~\ref{eq:updated_neuron_model}.
The input current is changed with the Postsynaptic current state, 
which is calculated in equation~\ref{eq:discrete_synapse}.

\begin{equation}
u_i[t] = \beta u_i[t-1] + (1 - \beta) g_i[t] 
\label{eq:updated_neuron_model}
\end{equation}


Current implementation of the model is dense. 
The sparsity can be added later.






% NOTES: Not related to the report, 
%Implementation steps
% 1- apply reset from previous spike, 
% the membrane voltage changes with the spike being 1 or 0
% u[t] = u[t-1] - s[t-1] * θ
% So if there is no spike in the previous one, no decrease
% This way there is no conditional check (branching) that will 
% make implementation harder. This is soft reset.

% After reset
% 2- Integrate Membrane voltage
% u[t] = β u[t] + (1 - β) i[t]
% new membrane potential current timestep.

% 3- Now the threshold check for the next step.
% s[t] = Θ(u[t] - θ)
% this will provide if the spike occured or not.
% And this will be used for later step.











% Below is latex template.
\section{Testing}

\begin{figure}[h]
\centering
\begin{lstlisting}
 Example Code
\end{lstlisting}
\caption{Transformation function definitons}
\label{fig:transformation_defs}
\end{figure}


\graphicspath{{images}}
\begin{figure}[H]
  \centering
  %\includegraphics[width=0.7\textwidth]{Pareto_of_default_image_provided.png}
  \caption{Pareto graph of different image transformation techniques}
  \label{fig:pareto_graph1}
\end{figure}



\begin{table}[htbp]
\centering
\caption{Power saving statistics for each transformation}
\label{tab:power_saving_stats}
\begin{tabular}{lrrr}
\toprule
\textbf{Transform} &
\textbf{Mean PS (\%)} &
\textbf{Min PS (\%)} &
\textbf{Max PS (\%)} \\
\midrule
Bright scale & 17.18 & 16.85 & 18.46 \\
Histogram    & -5.53 & -135.09 & 30.56 \\
Hungry blue  & 6.43 & 2.78 & 20.24 \\
\bottomrule
\end{tabular}
\end{table}

% Place this where you want the bibliography to appear
\bibliographystyle{plain} % Styles: plain, unsrt, alpha, etc.
\bibliography{references}

\end{document}